
\documentclass[final,a4paper]{article}
\usepackage[ngerman]{babel}
\usepackage[utf8]{inputenc}
\usepackage[T1]{fontenc}
\usepackage{lmodern}
\usepackage{listings}

\begin{document}
\lstset{tabsize=4}
\lstset{basicstyle=\small}
\lstset{language=java}

{\huge Programmiervorkurs Tag 3 - Aufgaben}

\bigskip

\begin{enumerate}
  \item{
	Schreibe ein Programm, das 23 mal "`Hello World!"' ausgibt. Verwende dazu eine Schleife.
	}
  \item{
	Schreibe ein Programm, das "`10 9 8 7 6 5 4 3 2 1 0 Bumm 0 1 2 3 4 5 6 7 8 9 10"' ausgibt.
	Verwende dazu 2 Schleifen, Arrays werden nicht benötigt.
	}
  \item{
	Im Array $notenspiegel$ sind die Noten einer Klausur gespeichert. Das erste Feld (Index $0$)
	enthält die Anzahl der Studenten, die eine 1 geschrieben haben, das zweite Feld enthält
	die Anzahl der Studenten, die eine 2 geschrieben haben usw.
	\begin{lstlisting}
	int[] notenspiegel = new int[] { 2, 5, 12, 8, 5 };
	\end{lstlisting}
	Hier haben also 2 Studenten eine 1, 5 Studenten haben eine 2 und 5 Studenten haben nicht bestanden (eine 5).
	Berechne den Notendurchschnitt. Verwende dazu eine Schleife, in der du auf die Arrayfelder zugreifst.
	}
  \item{
	Schreibe ein Programm, das ausgibt, wie viele Tage ein Monat hat. Erstelle dazu ein Array,
	das die Monatsnamen enthält, sowie ein weiteres Array, das in der selben Reihenfolge die Anzahl
	der Tage in den jeweiligen Monaten enthält. Schaltjahre sollen hier außer Acht gelassen werden.
	\begin{lstlisting}
	String[] namen = new String[] { "Januar", "Februar", ... }
	int[] tage = new int[] { 31, 28, ... }
	\end{lstlisting}
	Mit Hilfe einer Schleife soll jetzt für jeden Monat ausgegeben werden, wieviele Tage dieser hat.
	Für den Monat Februar soll also z.B. "`Der Monat Februar hat 28 Tage."' ausgegeben werden.
	}
  \item{
	Das Array $zahlenfolge$ soll um die Zahlen $108$, $540$ und $7200$ erweitert werden.
	\begin{lstlisting}
	int[] zahlenfolge = new int[] { 4, 8, 15, 16, 23, 42 };
	\end{lstlisting}
	(Hinweis: Dazu muss ein neues Array angelegt werden und die Zahlen in dieses übertragen werden.)
	Gib anschließend alle Elemente des neuen Arrays aus.
	}
  \item{
	Die Zahlenfolge $1, 1, 2, 3, 5, 8, 13, \ldots$ nennt man Fibonacci-Zahlen. Dabei entspricht jede Zahl
	in der Folge gerade der Summe ihrer beiden Vorgänger: $1+1=2, 1+2=3, 2+3=5, 3+5=8, \ldots$

	Schreibe eine Schleife, die alle Fibonacci-Zahlen ausgibt, die kleiner als $1000$ sind.
	}
\end{enumerate}

\end{document}

