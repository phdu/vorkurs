%\documentclass[pdftex,bibtotoc,liststotoc]{beamer}
\documentclass[pdftex,bibtotoc,liststotoc,handout]{beamer}
%\usetheme{Boadilla}
\usetheme[width=1.9cm]{PaloAlto}
%\usepackage{ngerman}
\usepackage{pgf}
\usepackage{graphicx}
\usepackage[ngerman]{babel}
\usepackage{lmodern}
\usepackage[babel,german=quotes]{csquotes}
\usepackage{color}

\usepackage{amsmath}
\usepackage{amsfonts}
\usepackage{latexsym}


\usepackage{listings}

\lstset{language=Java, 
	%basicstyle=\tiny\ttfamily, 
   basicstyle=\footnotesize\ttfamily, 
   keywordstyle=\color{blue!80!black!100}\ttfamily\bf, 
   identifierstyle=, 
   commentstyle=\color{green!50!black!100}\ttfamily, 
   stringstyle=\ttfamily\color{red}\ttfamily, 
   breaklines=true, 
   numbers=none, 
   numberstyle=\small, 
   frame=single, 
   backgroundcolor=\color{blue!3},
   showstringspaces=false,
   captionpos=b,
   keepspaces=true 
} 

%keine navigationssymbole unten rechts
%\setbeamertemplate{navigation symbols}{} 


% items enclosed in square brackets are optional; explanation below
\title[]{Methoden/Funktionen}
\subtitle[]{Oder: Ordnung ins Programm}
\author[]{Michael Merk}
\institute[]{
  Fachschaft Informatik\\
  HS Karlsruhe\\
  \texttt{kontakt@fachschaft-hska.de}
}
\date{\today}

%\logo{\pgfimage[width=2cm,height=2cm]{Hochschule_Karlsruhe_300}}
\titlegraphic{\includegraphics[scale=0.5]{Hochschule_Karlsruhe_300.jpg}}

\begin{document}

\begin{frame}[plain]
  \titlepage
\end{frame}

\begin{frame}[shrink=10]
\frametitle{\"Ubersicht}
\tableofcontents
\end{frame}

\section{Einf�hrung}
\subsection{Problemstellung}
\begin{frame}[fragile]
\frametitle{Problemstellung}
Was ist an folgendem Code problematisch?

\begin{lstlisting}
public class Main {
  public static void Main() {
    for(int i = 1; i <= 10; i++) {
      Console.WriteLine("Hello World");
    }
    //weiterer Code
    for(int i = 1; i <= 10; i++) {
      Console.WriteLine("Hello World");
    }
  }
}
\end{lstlisting}
\end{frame}

\begin{frame}[fragile]
\frametitle{Problemstellung}
\begin{lstlisting}[basicstyle=\tiny\ttfamily]
public class Main {
  public static void Main() {
    for(int i = 1; i <= 10; i++) {
      Console.WriteLine("Hello World");
    }
    //weiterer Code
    for(int i = 1; i <= 10; i++) {
      Console.WriteLine("Hello World");
    }
  }
}
\end{lstlisting}
Doppelter Code ist immer problematisch.
\begin{itemize}
\item Viel Code
\item �nderungen sind aufwendig
\item Code ist sonst nirgendwo verwendbar
\item Un�bersichtlich
\end{itemize}

\end{frame}

\subsection{L�sung}
\begin{frame}[fragile]
\frametitle{L�sung}
\begin{lstlisting}[basicstyle=\tiny\ttfamily]
public class Main {
  public static void Main() {
    for(int i = 1; i <= 10; i++) {
      Console.WriteLine("Hello World");
    }
    //weiterer Code
    for(int i = 1; i <= 10; i++) {
      Console.WriteLine("Hello World");
    }
  }
}
\end{lstlisting}
Doppelten Code an eine Stelle auslagern, wo er beliebig verwendet werden kann!
$\Rightarrow$ Methoden\\\medskip
\pause
\begin{block}{Methode}
Ein Unterprogramm, das beliebig oft ausgef�hrt werden kann.
\end{block}
\end{frame}


\begin{frame}[fragile]
\frametitle{L�sung}
\begin{lstlisting}
public class Main {
  public static void GibHelloWorldAus(){
  	for(int i = 1; i <= 10; i++) {
      Console.WriteLine("Hello World");
    }
  }

  public static void Main() {
    GibHelloWorldAus();
    //weiterer Code
    GibHelloWorldAus();
  }
}
\end{lstlisting}
\end{frame}


\section{Methoden}
\subsection{Parameter}

\begin{frame}[fragile]
\frametitle{Parameter}
Der Code ist immernoch nicht toll. Ich m�chte selber entscheiden, wie oft ich \lstinline|"Hello World"| ausgebe.\\\medskip
\pause
L�sung: Anzahl der Methode mitgeben $\Rightarrow$ Parameter
\begin{block}{Bei der Methode}
Der Parameter bei der Methode steht in der Klammer und besteht aus dem \textcolor{red}{Datentyp} und einem \textcolor{red}{Namen}\\
 $\Rightarrow$ \lstinline|public static void gibHelloWorldAus(int anzahl) {...}| 
\end{block}
\pause
\begin{block}{Beim Methodenaufruf}
Der Parameter steht ebenfalls in der Klammer und besteht aus einem konkreten Wert (des festgelegten Typs!)\\
 $\Rightarrow$ \lstinline|GibHelloWorldAus(5);| 
\end{block}
\end{frame}


\subsection{Beispiele}

\begin{frame}[fragile]
\frametitle{Beispiel 1}
\begin{lstlisting}
public class Main {
  public static void GibHelloWorldAus(int anzahl){
  	for(int i = 1; i <= anzahl; i++) {
      Console.WriteLine("Hello World");
    }
  }

  public static void Main() {
    GibHelloWorldAus(12);
    //weiterer Code
    GibHelloWorldAus(8);
  }
}
\end{lstlisting}
\end{frame}


\begin{frame}[fragile]
\frametitle{Beispiel 2}
Es kann beliebig viele Parameter geben. Diese werden mit einem Komma getrennt.
\begin{lstlisting}
public class Main {
  public static void GibStringAus(int anzahl,string ausgabeString){
  
  	for(int i = 1; i <= anzahl; i++) {
      Console.WriteLine(ausgabeString);
    }
  }

  public static void Main() {
    GibStringAus(5,"Hello World");
    //weiterer Code
    GibStringAus(6,"Goodbye World");
  }
}
\end{lstlisting}
\end{frame}




\subsection{Methoden sind Funktionen}

\begin{frame}
\frametitle{Methoden sind Funktionen}
 Hat das nicht was mit Mathe zu tun?\\\medskip
 Ja, Methoden in Java/C\# arbeiten genauso, wie Funktionen in der Mathematik.\medskip

\begin{block}{Funktion}
Eine Funktion nimmt Werte entgegegen, \glqq macht etwas damit\grqq und gibt dies dann zur�ck.
\end{block}
\begin{exampleblock}{Beispiel}
\[f(x)=x^2\]\\
Nimmt eine Zahl, quadriert sie und gibt das Ergebnis zur�ck.
\end{exampleblock}
\end{frame}

\subsection{Wertebereiche}
\begin{frame}[fragile]
\frametitle{Wertebereiche}
\begin{alertblock}{Wichtig}
Welchen Datentyp haben die Werte, die in eine Funktion reingehen?\\
Welchen Datentyp haben die Werte, die aus der Funktion rauskommen?\\
$\Rightarrow$ Muss man vorher wissen!
\end{alertblock}
\pause
\begin{exampleblock}{Beispiel}
\[f(x)=x^2\]\\
\lstinline|double|-Werte gehen in die Funktion rein und \lstinline|double|-Werte kommen raus.
\end{exampleblock}

\end{frame}


\subsection{Umsetzung}
\begin{frame}[fragile,shrink=10]
\frametitle{Umsetzung}
\begin{block}{Aufbau}
	\begin{lstlisting}
	public static R�ckgabetyp Name(Parametertyp Parametername){
	  //hier wird was gemacht
	  return ergebnis; //WICHTIG
	}
	\end{lstlisting}
\end{block}
\pause
\begin{lstlisting}
public class Main {
  public static double Quadriere(double x){
    return x*x;
  }
  
  public static void Main(){
    double ergebnis;
    
    for(int i=0;i<=20;i++){
    	ergebnis = Quadriere((double) i);
    	Console.WriteLine(ergebnis);
    }
  }
\end{lstlisting}
\end{frame}

\subsection{void}
\begin{frame}[fragile]
\frametitle{void}
\begin{block}{void}
\lstinline|void| bedeutet, dass eine Methode \glqq nichts\grqq zur�ck gibt.\\
$\Rightarrow$ es wird dann auch kein \lstinline|return| ben�tigt.
\end{block}
\begin{exampleblock}{Beispiel: Main-Methode}
\lstinline|public static void Main(){...}|\\
Die Main-Methode ist eine normale Methode, die aber auch als Einstiegspunkt ins Programm dient.
Die Main-Methode \glqq macht\grqq zwar was, gibt allerdings kein Ergebnis zur�ck.
\end{exampleblock}
\end{frame}


\section{Klassenvariablen}


\subsection{Probleme}
\begin{frame}[fragile]
\frametitle{H�ufige Probleme}
\begin{itemize}
	\item Methoden k�nnen nur einen Wert zur�ckgeben
	\item Methoden sollen auf gemeinsame Daten zugreifen\\(wird in den �bungen gebraucht!)
        
\end{itemize}
\medskip
L�sung\\
$\Rightarrow$ Klassenvariablen


\end{frame}

\subsection{Beispiel}
\begin{frame}[fragile]
\frametitle{Beispiel}
\begin{lstlisting}
public class Zaehler {
  public static int zaehlerStand = 0;
  
  public static void Erhoehe(){
    zaehlerStand++;
  }
  public static void Erniedrige(){
    zaehlerStand--;
  }
}
\end{lstlisting}
\medskip
Beide Methoden greifen auf den Zaehlerstand zu.

\end{frame}


\subsection{Deklaration}
\begin{frame}[fragile]
\frametitle{Deklaration von Klassenvariablen}
\begin{block}{Aufbau}
	\begin{lstlisting}
	public class Klasse{
	  //die Klassenvariable wird nach der Klasse
	  //nach der Klasse definiert
	  public static Typ name;
	  
	  public static void Main(){...}
	}
	\end{lstlisting}
\end{block}

\begin{alertblock}{ABER}
Statische Klassenvariablen mit Bedacht verwenden.\\
Es kann sehr schnell un�bersichtlich werden oder zu Seiteneffekten f�hren, wenn mehrere Methoden dieselbe Variable benutzen.
\end{alertblock}
\end{frame}








\section{Objektorientierung}
\begin{frame}
\textbf{Objektorientierung: eine kleine Einf�hrung}\\
Essentiell im Informatik-Studium\\ (f�r die �bungen nicht so wichtig).
\end{frame}

\begin{frame}
\frametitle{Objektorientierung}
\begin{block}{Frage}
Was wollen wir in der Programmierung h�ufig erreichen?\\
\pause
Antwort: Die reale Welt modellieren.
\end{block}
\medskip
\pause

Die Welt besteht aus Objekten.\\
\pause
\begin{block}{Objekte}
Objekte haben Eigenschaften (L�nge, Breite, H�he, Farbe,...)\\
Objekte haben Verhalten (leuchten, fahren, fliegen, reden, schlafen,...).
\end{block}
\end{frame}

\begin{frame}[fragile]
\frametitle{Beispiel}
\begin{lstlisting}
public class Main {
  
  public static void Main() {
    //Objekt anlegen
    Auto auto = new Auto();
    
    while(!auto.IsTankLeer()){
      auto.Fahre();
    }
  }
}
\end{lstlisting}
\end{frame}

\begin{frame}[fragile]
\frametitle{Beispiel}
Die zugeh�rige Auto-Klasse.
\begin{lstlisting}
public class Auto {
  int tankInhalt = 100;
  
  public boolean IsTankLeer(){
    if(tankInhalt > 0){
      return false;
    }else{
      return true;
    }
  }
  
  public void Fahre(){
    tankInhalt = tankInhalt - 1;
  }
}
\end{lstlisting}
Klasse anlegen: rechte Maustaste aufs Projekt $\Rightarrow$ neue Klasse
\end{frame}

\section{Literatur}
\begin{frame}
\frametitle{Literatur}
\begin{block}{Java}
\begin{thebibliography}{1}

\bibitem{javaInsel} Christian Ullenboom: \enquote{Java ist auch eine Insel}\\
										\small{\url{http://openbook.galileocomputing.de/javainsel/}}
			
\end{thebibliography}
\end{block}
\begin{block}{C\#}
\begin{thebibliography}{1}

\bibitem{guide} Golo Roden: \enquote{Auf der F�hrte von C\#}\\
										\url{http://www.guidetocsharp.de/}
\end{thebibliography}
\end{block}
\end{frame}


\section{Konventionen}
\begin{frame}
\frametitle{Kleiner Nachtrag zu Programmierkonventionen}
\begin{block}{Java}
Methoden beginnen klein, alle nachfolgenden Teilworte werden gro\ss geschrieben.\\
$\Rightarrow$ \lstinline|erhoeheZaehler()|\\
Genauso bei Variablennamen.
$\Rightarrow$ \lstinline|int zaehlerStand;|
\end{block}
\begin{block}{C\#}
Methoden beginnen gro\ss, alle nachfolgenden Teilworte werden gro\ss geschrieben.\\
$\Rightarrow$ \lstinline|ErhoeheZaehler()|\\
Variablennamen beginnen klein, alle nachfolgenden Teilworte werden gro\ss geschrieben.\\
$\Rightarrow$ \lstinline|int zaehlerStand;|
\end{block}
Wenn man sich nicht daran h�lt ist es kein Fehler, sieht nur manchmal unsch�n aus.
\end{frame}



\begin{frame}
\frametitle{ENDE}
\Huge{Viel Spa\ss  bei der �bung.\\Evaluieren nicht vergessen!}
\end{frame}

%--- TEST FRAMES -----------------%
%\section[Test]{Test}
%\subsection[Kurzform]{Sourcecode}
%\begin{frame}[fragile]
%\frametitle{Einbettung von Source}
%%\begin{semiverbatim}
%\pause
%\begin{columns}
%  \begin{column}{0.5\textwidth}
%\begin{lstlisting}
%public class Main {
%  public static void main() {
%    for(int i = 1; i <= 10; i++) {
%      System.out.println(i);
%    }
%    //weiterer Code
%    for(int i = 1; i <= 10; i++) {
%      System.out.println(i);
%    }
%  }
%}
%\end{lstlisting}
%%\end{semiverbatim}
%
%  \end{column}
%  \begin{column}{0.5\textwidth}
%  \begin{lstlisting}
%public class Main {
%  public static void main() {
%    for(int i = 1; i <= 10; i++) {
%      System.out.println(i);
%    }
%    //weiterer Code
%    for(int i = 1; i <= 10; i++) {
%      System.out.println(i);
%    }
%  }
%}
%\end{lstlisting}
%    \end{column}
%\end{columns}
%\end{frame}
%
%
%
%\subsection[Kurzform]{columns}
%\begin{frame}{Splitting a slide into columns}
%
%The line you are reading goes all the way across the slide.
%From the left margin to the right margin.  Now we are going
%the split the slide into two columns.
%\bigskip
%
%\begin{columns}
%  \begin{column}{0.5\textwidth}
%    Here is the first column.  We put an itemized list in it.
%    \begin{itemize}
%      \item This is an item
%      \item This is another item
%      \item Yet another item
%    \end{itemize}
%  \end{column}
%
%  \begin{column}{0.3\textwidth}
%    Here is the second column.  We will put a picture in it.
%    \centerline{\includegraphics[width=0.7\textwidth]{Hochschule_Karlsruhe_300}}
%  \end{column}
%\end{columns}
%\bigskip
%
%The line you are reading goes all the way across the slide.
%From the left margin to the right margin.
%
%\end{frame}
%
%
%
%\subsection[Kurzform]{Naaaame}
%
%\begin{frame}{A sample slide}
%
%\begin{theorem}[The Poincar\'e inequality]
%Suppose $\Omega\in\mathbf{R}^n$ is a bounded domain with smooth
%boundary.  Then there exists a $\lambda>0$, depending only on
%$\Omega$, such that for any function $f$ in the Sobolev space
%$H^1_0(\Omega)$ we have:
%
%\[
%  \int_\Omega |\nabla u|^2 \,dx \ge 
%  \lambda \int_\Omega |u|^2 \,dx .
%\]
%\end{theorem}
%
%Here is what \emph{itemized} and \emph{enumerated} lists look like:
%
%\begin{columns}
%  \begin{column}{0.45\textwidth}
%  \begin{itemize}
%    \item itemized item 1
%    \item itemized item 2
%    \item itemized item 3
%  \end{itemize}
%  \end{column}
%
%  \begin{column}{0.45\textwidth}
%  \begin{enumerate}
%    \item enumerated item 1
%    \item enumerated item 2
%    \item enumerated item 3
%  \end{enumerate}
%  \end{column}
%\end{columns}
%
%\end{frame}
%
%%zum kopieren
%\begin{frame}
%\begin{columns}
% \begin{column}{0.5\textwidth}
%   \end{column}
%  \begin{column}{0.5\textwidth}
%  \end{column}
%  \end{columns}
%\end{frame}
%
%

\end{document}