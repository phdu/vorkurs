\documentclass[final]{beamer}
\usepackage[ngerman]{babel}
\usepackage[utf8]{inputenc}
\usepackage[T1]{fontenc}
\usepackage{lmodern}
\usepackage{listings}
\usepackage{graphicx}
\usepackage{color}
\usepackage{amssymb}
\newcommand{\ThemeFolder}{./FSIBeamerTheme}
\RequirePackage{\ThemeFolder/beamerthemeFSI}

\DeclareGraphicsExtensions{.pdf,.png}

\mode<presentation>

\newcommand{\minutenMittagspause}{60}

\title{Programmiervorkurs für Erstsemester}

%
%\setbeamertemplate{title page}{
%  \begin{center}
%	{\usebeamerfont{title}\usebeamercolor[fg]{title}\inserttitle}
%  \end{center}
%

\setbeamertemplate{title page}{
	\begin{center}
		\color{FSIblue}
			\resizebox{\textwidth}{!}{Programmiervorkurs}\\
			\vspace{0.3\baselineskip}
			\huge{Einführung in Java}\\
			\huge{Tag 2}
			\vfill
			\large{Moritz Grimm}\\
			\tiny{SS2015}
	\end{center}
}

\begin{document}
\shorthandoff{"}

\lstset{
	backgroundcolor=\color{white},
	basicstyle=\color{black}\small\ttfamily,
	breakatwhitespace=true,
	breaklines=true,
	commentstyle=\color{green},
	extendedchars=true,
	identifierstyle=\color{cyan},
	inputencoding=utf8,
	keywordstyle=\color{orange},
	language=java,
	literate={€}{{\euro}}1 {ä}{{\"a}}1 {ö}{{\"o}}1 {ü}{{\"u}}1,
	numbers=left,
	numbersep=5pt,
	numberstyle=\tiny,
	showstringspaces=true,
	stringstyle=\color{red},
	tabsize=3
}

\begin{frame}
	\titlepage
\end{frame}

\section{Ablauf}
\begin{frame}
	\frametitle{Ablauf}
	\begin{itemize}
		\item{09:30 Vorstellung der Lösungen des Vortages}
		\item{ab 10:00 Vorlesung}
		\item{\minutenMittagspause\ Minuten Mittagspause}
		\item{gegen 12:30 Übungen}
	\end{itemize}
\end{frame}

\begin{frame}
	\frametitle{Inhaltsübersicht Vorkurs}
	\begin{itemize}
	{\color{gray}
		\item {Tag 1: Variablen, Datentypen, Konvertierungen, Arithmetik, Netbeans, Einführung Debugging}
		{\color{black}
		\item {Tag 2: Boolesche Ausdrücke, Kommentare, If-Abfragen, Switch-Case}
		}
		\item {Tag 3: Arrays, (Do-)While-Schleife, For-Schleifen, Weiterführung Debugging}
		\item {Tag 4: (statische) Methoden, Klassenvariablen, JavaDoc, Exceptions}
	}
	\end{itemize}
\end{frame}

\begin{frame}
	\frametitle{Inhaltsübersicht Tag 2}
	\begin{itemize}
		\item {Kommentare}
		\item {Boolesche Ausdrücke}
		\item {If-Abfragen}
		\item {Switch-Case}
	\end{itemize}
\end{frame}

\section{Kommentare}
\begin{frame}
	\frametitle{Kommentare}
	\begin{itemize}
		\item{erleichtern das Verständnis des Quelltextes}
		\item{haben keinen Einfluss auf den Programmablauf}
		\item{Programmdokumentation durch JavaDoc}
	\end{itemize}
\end{frame}

\subsection{Arten}
\begin{frame}[containsverbatim]
	\begin{lstlisting}
JavaDoc:
/**
 * Kommentar (auch ueber mehrere Zeilen),
 * der automatisch zu html-Dokumentation
 * verarbeitet werden kann
 */

Blockkommentar:
/*
 * Mehrzeilige Kommentare sind ideal, wenn
 * viele Informationen unterzubringen sind.
 * Es gilt die Devise: so knapp wie
 * moeglich, so ausfuehrlich wie noetig.
 */

Zeilenkommentar:
// endet mit Zeilenumbruch
	\end{lstlisting}
\end{frame}

\subsection{Verwendung}
\begin{frame}
	\frametitle{Verwendung von Kommentaren}
	\begin{itemize}
		\item{rettet Leben! Besonders das eigene!}
		\item{geben nachfolgenden Entwicklern Hinweise, wie der Quelltext zu verstehen ist}
		\item{sollten auf einem höheren Abstraktionslevel beschreiben was passiert}
		% Beispiel - über if ( a == b ) { nicht "a und b werden überprüft" sondern "hier sichergestellt, dass ... Fluxkompensator..."
		\item{sehr praktisch als Gedächtnisstütze: TODOs setzen}
		\item{zum Testen können Teile des Quellcodes zeitweise auskommentiert werden}
		% auch wenn es gerade am Anfang oft verlockend ist zu beschreiben was die einzelnen Codeschnipsel machen
		\pause
		\item{ein Stück weit auch eine "Religionsfrage"}
	\end{itemize}
\end{frame}

\subsection{Beispiel}
\begin{frame}[containsverbatim]
\frametitle{Beispiel}
	\begin{lstlisting}
/**
 * This Class is dangerous!
 * Do not execute it.
 * @author memo
 */
public class GladOS {
	/*
	 * Die wichtigste Methode:
	 * alles was ausgefuehrt wird muss
	 * zumindest hier aufgerufen werden.
	 */
	public static void main(String[] args){
		// the cake is a lie
		boolean cake = true;
	}
}
	\end{lstlisting}
%/**
% * Ein Hello World Programm in Java
% *
% * @author memo
% * @version 1.0
% */
%public class Hello {
%
%	public static void main(String[] args){
%		//gibt Hello World! aus
%		System.out.println("Hello World!");
%		/*
%		 * soll noch ganz viele andere
%		 * tolle sachen ausgeben
%		 */
%	}
%}
\end{frame}

\begin{frame}[containsverbatim]
	\frametitle{Deklaration einer Methode mit Javadoc.}
	\begin{lstlisting}
/**
 * determines whether this shape is
 * located at point pos.
 *
 * @param pos the point that is queried.
 * @return whether this shape is located
 * at point pos.
 */
public boolean amIHere(Point pos){
	/*
	 * important Voodoo is done here
	 */
}
	\end{lstlisting}
\end{frame}

\begin{frame}
	\frametitle{Verwendung der Methode mit JavaDoc.}
	\includegraphics[scale=0.5]{JavaDoc_example_2_1.png}
\end{frame}

\section{boolean}
\subsection{Wahrheitswerte}
\begin{frame}
	\frametitle{Boolsche Ausdrücke - Wahrheitswerte}
	\textbf{100 Fakten zu Boolschen Ausdrücken}
	\begin{itemize}
		\item{entweder wahr (1) oder falsch (0)}
		\item{oft das Ergebnis eines Vergleichs}
		\item{können kombiniert werden}
		\item{werden zur Entscheidungsfindung verwendet}
	\end{itemize}
\end{frame}

\subsection{Vergleiche}
\begin{frame}
	\frametitle{Boolsche Operatoren - Vergleiche}
	\begin{tabular}{c c c}
		\textbf{>}  & - & größer \\&&\\
		\textbf{<} & - & kleiner \\&&\\
		\textbf{>=} & - & größer gleich \\&&\\
		\textbf{<=}  & - &  kleiner gleich \\&&\\
		\textbf{==} & - & gleich \\&&\\
		\textbf{!=} & - & ungleich \\
	\end{tabular}
\end{frame}

\subsection{Verknüpfungen}
\begin{frame}
	\frametitle{Boolsche Operatoren - Verknüpfungen}
	\begin{tabular}{c c c}
		\textbf{\&}  & - & und \\&&\\
		\textbf{\^\ }  & - &  exklusives oder \\&&\\
		\textbf{|} & - & oder \\&&\\
		\textbf{!} & - & nicht \\&&\\
	\end{tabular}
\end{frame}

\begin{frame}
	\frametitle{Wahrheitstabellen}
	\center{0 - false, 1 - true} \\
	\begin{center}
	\begin{tabular}{c  c c}
		\begin{tabular}{c c c c c | c}
			& \textbf{a}  & & \textbf{b} & & \textbf{\&} \\
			\hline
			& 0 & & 0 & &  0\\
			& 0 & & 1 & &  0\\
			& 1 & & 0 & &  0\\
			& 1 & & 1 & &  1\\
		\end{tabular}
		& &
		\begin{tabular}{c c c c c | c}
			& \textbf{a}  & & \textbf{b} & & \textbf{|} \\
			\hline
			& 0 & & 0 & &  0\\
			& 0 & & 1 & &  1\\
			& 1 & & 0 & &  1\\
			& 1 & & 1 & &  1\\
		\end{tabular}
		\\ &\\ &\\
		\begin{tabular}{c c c c c | c}
			& \textbf{a}  & & \textbf{b} & & \textbf{\^} \\
			\hline
			& 0 & & 0 & &  0\\
			& 0 & & 1 & &  1\\
			& 1 & & 0 & &  1\\
			& 1 & & 1 & &  0\\
		\end{tabular}
		& &
		\begin{tabular}{c c c | c}
			& \textbf{a} & & \textbf{!} \\
			\hline
			& 0 & & 1\\
			& 1 & & 0\\
		\end{tabular}
		\\
		\end{tabular}
	\end{center}
\end{frame}

\begin{frame}
	\frametitle{Rangfolge der Operatoren}
	\center{Sortiert nach absteigender Bindungsstärke}\\
	\begin{center}
	\begin{tabular}{|l c | l|}
		\hline
		++, -- --  & &  Inkrement und Dekrement \\
		\hline
		! & & Negation \\ \hline

		*, /, \% & & Multiplikation, Division, Modulo \\ \hline
		+, - & & Addition und Subtraktion \\ \hline
		==, != & & Vergleich \\ \hline
		\& & & Und \\ \hline
		| & & Oder \\ \hline
		\&\& & & Kurzschlussoperator Und \\ \hline
		|| & & Kurzschlussoperator Oder \\ \hline
		= & & Zuweisung \\
		\hline
	\end{tabular}
	\center{Unäre Operatoren => Standard Rechenzeichen => binäre Operatoren => Zuweisungen}
	\end{center}
\end{frame}

\subsection{boolscher Ausdruck}
\begin{frame}
	\frametitle{Definition: Boolscher Ausdruck}
	\begin{itemize}
		\item{Wahrheitswert - kann wahr oder falsch sein}
		\item{werden meist als Bedingungen verwendet}
		\item{mehrere Werte können verknüpft werden - zu sogenanneten boolschen Ausdrücken}
		\item{"StudentIn an der Hochschule" \&\& "Ist im ersten Semester" (kann der Compiler so nicht auswerten, Menschen schon)}
		\pause
		\item{boolean studiesAtHsKa}
		\item{int semester}
		\pause
		\item{studiesAtHsKa \&\& semester == 1}
	\end{itemize}
\end{frame}

\subsection{Beispiel}
\begin{frame}
	\frametitle{Beispiel: \\Schreibe einen boolschen Ausdruck in Java Syntax, der einen Sportwagen erkennt}
	Unsere Definition eines Sportwagens:	\\
	\begin{itemize}
		\item{maximal zwei Türen, keine Rücksitze, außerdem:}
		\item{Höchstgeschwindigkeit von mindestens 200 km/h und  Mindestbeschleunigung von 0 auf 100 km/h in 8 Sekunden}
		\item{oder Höchstgeschwindigkeit von mindestens 280 km/h und mindestens 250 PS}
	\end{itemize}
\end{frame}

\begin{frame}[containsverbatim]
	\frametitle{Auszuwertende Variablen}
	\begin{lstlisting}
// true, wenn das Auto Ruecksitze hat
boolean hasBackseats;
int doors; // Anzahl Tueren
// Beschleunigung von 0 auf 100 in Sekunden
double acceleration0To100InS;
// Hoechstgeschwindigkeit in KM/H
int maximumSpeedInKmPerH;
// Leistung in PS
int powerInPS;
	\end{lstlisting}
\end{frame}
%Beispiele für die Verwendung der Syntax

\begin{frame}[containsverbatim]
	\frametitle{Lösung}
\begin{lstlisting}
doors < 3
&& !hasBackseats
&& ((maximumSpeedInKmPerH > 200 && acceleration0To100InS < 8)
	|| (maximumSpeedInKmPerH > 280 && powerInPS > 250))
\end{lstlisting}
\vfill
ODER
\vfill
\begin{lstlisting}
doors <= 2
&& hasBackseats == false
&& ((maximumSpeedInKmPerH > 200 && acceleration0To100InS < 8)
	||(maximumSpeedInKmPerH > 280 && powerInPS > 250))
\end{lstlisting}

\end{frame}

\subsection{Operatoren}
\begin{frame}
	\frametitle{Kurzschlussoperatoren}
	\begin{itemize}
		\item{UND und ODER gibt es auch als sogenannte Kurzschlussoperatoren: \&\& und ||}
		\item{=> der Ausdruck wird nur solange ausgewertet, bis das Ergebnis feststeht}
	\end{itemize}
\end{frame}


\section{Gleitkommazahlen}
\begin{frame}
	\frametitle{Exkurs: Gleitkommazahlen}
	\begin{itemize}
		\item{Gleitkommazahlen in Java sind anfällig für Rundungsfehler}
		\item{schon der Variablentyp kann einen Unterschied machen}
		\item{niemals direkt (mit ==) vergleichen!}
	\end{itemize}
\end{frame}

\subsection{Vergleiche}
\begin{frame}
	\frametitle{Exkurs: Gleitkommazahlen}
	\begin{itemize}
		\item{0.02d (double) == 0.02f (float) ?}
		\item{=> um sicherzugehen lieber prüfen, ob die Abweichung nur minimal ist: \\
		Math.abs(0.02d - 0.02f) < 4.5E-10 ?}
		\item{Math.abs(zahl) berechnet den Betrag von zahl}
	\end{itemize}
\end{frame}

\subsection{Größenordungsprobleme}
\begin{frame}
	\begin{itemize}
		\item{Achtung beim Verrechnen von Zahlen, die einige Größenordungen auseinander liegen}
		\item{z.B. 100000000000000000.0 + 8 = 100000000000000000.0}
		\item{Grund: Gleitkommazahlen werden als Basis und Exponent gespeichert}
		\item{die Basis muss alle relevanten Teile der Zahl enthalten die dann vom Exponent verschoben werden können}
		\item{=> es können sehr große und sehr kleine Zahlen gespeichert werden, aber nicht beides}
		%TODO sinnvolles Beispiel mit einer verkleinerten Variante von Float (Basis 2 Ziffern, Exponent 1 Ziffer)
	\end{itemize}
\end{frame}

\section{if}
\subsection{Beispiel}
\begin{frame}[containsverbatim]
	\frametitle{Fallunterscheidung durch if-Abfragen}
	Anweisung wird nur dann ausgeführt, wenn eine bestimmte Bedingung erfüllt ist:
	\begin{lstlisting}
if (Bedingung) {
	// mach was
} else if (andere Bedingung){
	// mach was anderes
} else {
	// lass es bleiben
}
	\end{lstlisting}
	(else if und else optional)
\end{frame}

\begin{frame}[containsverbatim]
	\frametitle{Beispiel}
	\begin{lstlisting}at
if(weHaveMate){
	System.out.println
		("Nimm ne Mate und geh zur Vorlesung");
} else if (weHaveKaffee){
	System.out.println
		("Nimm nen Kaffee und geh zur Vorlesung");
} else {
	System.out.println
		("Viel Glueck da draussen - du bist auf dich alleine gestellt");
}
	\end{lstlisting}
\end{frame}

\subsection{Syntax}
\begin{frame}
	\frametitle{if-Abfragen - Syntax}
	\begin{itemize}
		\item{auf die Klammer, in der die Bedingung angegeben wird, folgt kein Semikolon!}
		\item{in einem If-Block können beliebig viele Anweisungen stehen.}
		\item{die einzelnen Blöcke werden durch geschweifte Klammern getrennt}
	\end{itemize}
\end{frame}

\begin{frame}[containsverbatim]
	\frametitle{noch ein etwas praktischeres Beispiel:}
	% evtl auch anderes bsp. da int und geteilt nicht gut?
	% Kein Problem, da das Ergebnis ja nicht gespeichert sondern ausgegeben wird!
	\begin{lstlisting}
int a;
double b;
\\ ...
\\ hier muessen den Variablen noch
\\ Werte zugewiesen werden
\\ ...
if (a != 0) {
	System.out.print("b/a: ");
	System.out.println(b/a);
} else {
	System.out.println("Division durch 0!");
}
	\end{lstlisting}
	Hier ist die Bedingung das Ergebnis eines Vergleichs.
\end{frame}

\subsection{false == true}
\begin{frame}[containsverbatim]
	\frametitle{Abkürzungen bei boolschen Variablen}
	\begin{lstlisting}
boolean isStudent = true;
if (isStudent) {
	// gewaehre Studentenrabatt
}
//entspricht
if (isStudent == true) { ... }

//umgedreht:
if (!isStudent){
	//gewaehre keinen Studentenrabatt
}
//entspricht
if (isStudent == false) { ... }
	\end{lstlisting}
\end{frame}

\subsection{Verschachtelung}
\begin{frame}[containsverbatim]
	\frametitle{Verschachtelte if-Abfragen}
	\begin{lstlisting}
int age;
...
if(age < 18){
	if(age < 12){
		System.out.println("kein Eintritt");
	} else { // if 12 < age < 18 ;)
		System.out.println("ermaessigter Eintritt");
	}
} else {
	System.out.println("voller Eintritt");
}
	\end{lstlisting}
	=> Vorsicht! Wird leicht unübersichtlich!
\end{frame}

\section{switch-case}
\subsection{Überblick}
\begin{frame}[containsverbatim]
	\frametitle{Fallunterscheidung durch Switch/Case}
	\begin{itemize}
		\item{Fallunterscheidung in Abhängigkeit von einer Variablen}
		\item{nur ganzzahlige Typen oder char}
		\item{jeder Wert darf nur einmal vorkommen}
		\item{läuft durch bis break oder bis zum Ende des Switch}
		\item{trifft keiner der beachteten Fälle ein wird (ähnlich dem else) ein default ausgeführt, wenn vorhanden}
	\end{itemize}
\end{frame}


\subsection{Beispiele}
\begin{frame}[containsverbatim]
	\frametitle{Beispiel}
	\begin{lstlisting}
int fallThrough;
...
switch (fallThrough) {
case 0:
	System.out.println("Erfolgreich");
	break;
case 1:
	System.out.println("Abgebrochen");
	// fall through
case 2:
	System.out.println("Gescheitert");
	break;
default:
	System.out.println("Unbekannt");
}
	\end{lstlisting}
\end{frame}

\begin{frame}[containsverbatim]
\frametitle{switch/case verglichen mit if/else}
	\begin{lstlisting}
char a;
// input als Name waere besser
// aber das passt vom Layout nicht
	\end{lstlisting}
	\begin{tabular}{c c c c}
	\begin{lstlisting}
...
switch (a) {
case 'r':
	// lese
	break;
case 'q':
	// beenden
	break;
case 'n':
	// Neustart
	break;
}
	\end{lstlisting}

& & &

	\begin{lstlisting}
...
if (a == 'r') {
	// lese Eingabe
} else if (a == 'q') {
	// beenden
} else if (a == 'n') {
	// Neustart
}
	\end{lstlisting}\\
	\end{tabular}
\end{frame}

\section{weitere Planung}
\begin{frame}
\frametitle{weitere Planung}
	\begin{itemize}
		\item{weiter gehts in ca. \minutenMittagspause\ Minuten}
		\item{heute Nachmittag bleiben wir da bis ihr fertig seid, es gibt also kein festes Ende bis zu dem ihr fertig sein müsst}
		\item{morgen früh ab 9:30 Uhr findet die Besprechung der Aufgaben von heute Nachmittag statt}
		\item{danach (ca. ab 10:00 Uhr) beginnt die nächste Vorlesung. Themen: Arrays und Schleifen}
	\end{itemize}
\end{frame}

\section{Quellen \& Lizenz}
\begin{frame}
	\frametitle{Quellen und Lizenz}
	\begin{center}
		\includegraphics[width=250px]{gfx/fsi}
	\end{center}
	\begin{itemize}
		\item{Original von Anna Roes}
		\item{Überarbeitet 2011 \& 2015 von Moritz Grimm}
		\item{Überarbeitet 2013 von Sebastian Wörner}
	\end{itemize}
\end{frame}

\end{document}
