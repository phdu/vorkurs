\documentclass[final,a4paper]{article}
\usepackage[ngerman]{babel}
\usepackage[utf8]{inputenc}
\usepackage[T1]{fontenc}
\usepackage{lmodern}
\usepackage{listings}
\usepackage{graphicx}
\usepackage{xcolor}
\usepackage{eurosym}
\usepackage{enumerate}

\begin{document}
\lstset{
	breakatwhitespace=true,
	breaklines=true,
	extendedchars=true,
	inputencoding=utf8,
	language=java,
	literate={€}{{\euro}}1 {ä}{{\"a}}1 {ö}{{\"o}}1 {ü}{{\"u}}1,
	numbers=none,
	numbersep=5pt,
	numberstyle=\tiny,
	showstringspaces=true,
	tabsize=3,
%	backgroundcolor=\color{lightgray},
	basicstyle=\color{black}\small\ttfamily,
	commentstyle=\color{gray},
	identifierstyle=\color{cyan},
	keywordstyle=\color{orange},
	stringstyle=\color{purple}
}

{\huge Programmiervorkurs Tag 2 - Lösungen}

\bigskip

\section*{Aufgabe T2.1}
\begin{enumerate}[(a)]
	\item false
	\item true
	\item false
	\item false
	\item true
	\item true
	\item false
	\item true
	\item false
	\item false
	\item true
	\item true
	\item true
\end{enumerate}

\section*{Aufgabe T2.2}
\begin{enumerate}[(a)]
	\item stimmt
	\item stimmt
	\item stimmt nicht
	\item stimmt
	\item stimmt nicht
	\item stimmt nicht
	\item stimmt
\end{enumerate}


\section*{Aufgabe T2.3}
\begin{lstlisting}
(x > 25) && x % 2 != 0
\end{lstlisting}

\section*{Aufgabe T2.4}
\begin{lstlisting}
(jahr % 4 == 0 && jahr % 100 != 0) || jahr % 400 == 0
\end{lstlisting}

\section*{Aufgabe T2.5}
\begin{lstlisting}
public static void main(String[] args){
	boolean male = true;
	String name = "Peter";
	
	if (male) {
		System.out.println("Guten Tag, Herr " + name);
	else {
		System.out.println("Guten Tag, Frau " + name);
	}
}
\end{lstlisting}

\section*{Aufgabe T2.6}
\begin{lstlisting}
public static void main(String[] args){
	double p = 0.0;
	double q = 0.0;
	double diskriminante = Math.pow(p * 0.5, 2.0) - q;
	double x1;
	double x2;

	if (diskriminante < 0) {
		System.out.println("Für die gegebenen p und q existiert keine Loesung im Bereich der reellen Zahlen!");
	} else {
		x1 = p * -0.5 + Math.sqrt(diskriminante);
		x2 = p * -0.5 - Math.sqrt(diskriminante);
		System.out.println("x1: " + x1);
		System.out.println("x2: " + x2);
	}
}
\end{lstlisting}

\section*{Aufgabe T2.7}
\begin{lstlisting}
public static void main(String[] args){

	int winkel = 0;

	if (winkel >= 0 && winkel <= 45 || winkel > 315 && winkel < 360) {
		System.out.println("Norden");
	} else if (winkel > 45 && winkel <= 135) {
		System.out.println
	} else if (winkel > 135 && winkel <= 225) {
		System.out.println("Sueden");
	} else if (winkel > 225 && winkel <= 315) {
		System.out.println("Westen");
	} else {
		System.out.println("Ungueltiger Winkel! Bitte einen Wert								 zwischen 0 und 359 wählen!");
	}
}
\end{lstlisting}

\section*{Aufgabe T2.8}
\begin{lstlisting}
public static void main(String[] args){
	int winkel = 0;
	int normWinkel = ((winkel + 45) % 360) / 90;
	switch(normWinkel){
		case 0:
			System.out.println("Norden");
			break;
		case 1:
			System.out.println("Osten");
			break;
		case 2:
			System.out.println("Sueden");
			break;
		case 3:
			System.out.println("Westen");
			break;
	}
}
\end{lstlisting}

\end{document}