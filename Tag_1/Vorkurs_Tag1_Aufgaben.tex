\documentclass[final,a4paper]{article}
\usepackage[ngerman]{babel}
\usepackage[utf8]{inputenc}
\usepackage[T1]{fontenc}
\usepackage{lmodern}
\usepackage{listings}
\usepackage{graphicx}
\usepackage{xcolor}
\usepackage{eurosym}
\usepackage{enumerate}

\begin{document}
\lstset{
	breakatwhitespace=true,
	breaklines=true,
	extendedchars=true,
	inputencoding=utf8,
	language=java,
	literate={€}{{\euro}}1 {ä}{{\"a}}1 {ö}{{\"o}}1 {ü}{{\"u}}1,
	numbers=none,
	numbersep=5pt,
	numberstyle=\tiny,
	showstringspaces=true,
	tabsize=3,
%	backgroundcolor=\color{lightgray},
	basicstyle=\color{black}\small\ttfamily,
	commentstyle=\color{gray},
	identifierstyle=\color{cyan},
	keywordstyle=\color{orange},
	stringstyle=\color{purple}
}
\lstset{identifierstyle=\color{black}}


{\huge Programmiervorkurs Tag 1 - Aufgaben}

\bigskip

\textbf{\color{red}Denkt daran für jede Aufgabe ein neues Projekt anzulegen!}

Die Sterne bezeichnen die Schwierigkeit der Aufgaben.

\section*{Aufgabe T1.0 (Hello World)}
Erstelle auf dem IZ Laufwerk ein Verzeichnis.
Danach startest du NetBeans (oder Eclipse).
Du solltest für jede Übung ein neues Projekt in deinem Verzeichnis erstellen.
Das erste Übungsprogramm soll auf der Konsole den Text „Hello World“ ausgeben.

\section*{Aufgabe T1.1 *}
Das Programm soll den Text ausgeben\\
\texttt{„Ich bin im 1. Semester.“}\\
Wobei das Semester aus einer Variablen (z.B. int semester;) ausgegeben werden soll.

\section*{Aufgabe T1.2 **}
Schreibe ein Programm, das Meter in Fuß umrechnet. 

Das Programm soll auch Kommazahlen akzeptieren. 
Die Meter, die in Fuß umgerechnet werden sollen, stehen wieder in einer entsprechenden Variable. 
(Keine interaktive Eingabe.)

Ein Fuß (1 ft) = 30,48 cm.

\section*{Aufgabe T1.3 **}
Ein Mobiltarif kostet einen Grundbetrag (23,34 \euro) und je Minute 10 Cent. 

Schreibe ein Programm, welches für eine beliebige Anzahl von Minuten (die wieder mal in einer Variable stehen) den Gesamtbetrag berechnet.

\section*{Aufgabe T1.4 **}
Lege ein Programm an, in dem es zunächst folgende Integer Variablen gibt: einser, zweier, dreier, vierer, fuenfer

Diese fünf Variablen beinhalten jeweils die Anzahl der Studenten, die die entsprechende Note bei einer Klausur erreicht haben. Berechne nun aus diesen Noten den Notendurchschnitt der Klausur.

\section*{Aufgabe T1.5 **}
Welchen Wertetyp und welchen Wert hat das Ergebnis folgender Ausdrücke? 

Den Wert kannst du prüfen indem du den Ausdruck einfach von einem Programm ausgeben lässt. Die Lösung zu den Wertetypen werden morgen mit der Musterlösung besprochen.
\begin{enumerate}[(a)]
\item 5 / (int)2.0 * 3
\item (double)(3 / 2) / 2
\item ((int) (1.2 * 1.3)) * (short)10
\item 1 + 2.0 / 5.0f – 10
\end{enumerate}


\section*{Aufgabe T1.6 ***}
Schreibe ein Programm, das in einer Variablen days die Tage seit dem 1.1.1970 gespeichert hat.

Gebe nun auf der Konsole das Datum zu days aus. 

Zur Vereinfachung gehe davon aus, dass jeder Monat 30 Tage hat und jedes Jahr aus 12 Monaten besteht.
Beispiel:
\begin{itemize}
\item Die Variable days hat den Wert 0. Die Ausgabe soll sein: 1.1.1970
\item Die Variable days hat den Wert 1. Die Ausgabe soll sein: 2.1.1970
\item Die Variable days hat den Wert 29. Die Ausgabe soll sein: 30.1.1970
\item Die Variable days hat den Wert 30. Die Ausgabe soll sein: 1.2.1970
\item Die Variable days hat den Wert 360. Die Ausgabe soll sein: 1.1.1971
\item usw.
\end{itemize}

\section*{Aufgabe T1.7 **}
Was ist an den folgenden Zeilen falsch?
\begin{lstlisting}
int dieAntwort = zweiundvierzig;
int einkommenIn\euro = 1337;
int riesigeZahl = 5400000000;
int 42istdieAntwort = 23; 
int berechnung = "23" - 42;
\end{lstlisting}

\section*{Aufgabe T1.8 **}
Beschreibe in eigenen Worten was die folgende Funktion tut:
\begin{lstlisting}
public void main(){
	int studenten = 150;
	int faecher = 8;
	int durchfallquoteInProzent = 20;
	int ergebnis = studenten*faecher*durchfallquoteInProzent/100;
	System.out.println(ergebnis);
}
\end{lstlisting}

\end{document}