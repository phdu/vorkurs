\documentclass[final]{beamer}
\usepackage[ngerman]{babel}
\usepackage[utf8]{inputenc}
\usepackage[T1]{fontenc}
\usepackage{lmodern}
\usepackage{listings}
\usepackage{graphicx}
\usepackage{tikz}
\usepackage{xcolor}
\usepackage{xspace}
\newcommand{\ThemeFolder}{FSIBeamerTheme}
\RequirePackage{\ThemeFolder/beamerthemeFSI}

\setcounter{tocdepth}{1}
\usetikzlibrary{arrows,shapes}

\DeclareGraphicsExtensions{.pdf,.png}

\newcommand{\keyWord}[1]{\textbf{#1}}
\newcommand{\ra}{\ensuremath{\rightarrow}\xspace}


\mode<presentation>

\title{Programmiervorkurs für Erstsemester}

\setbeamertemplate{title page}{
  \begin{center}
    \color{FSIblue}
      \resizebox{\textwidth}{!}{Programmiervorkurs}\\
      \vspace{0.3\baselineskip}
      \huge{Einführung in Java}
      \vfill
      \tiny{WS 2013/14}
  \end{center}
}

\begin{document}

\lstset{tabsize=4}
\lstset{basicstyle=\small}
\lstset{language=java}

\begin{frame}
  
  \titlepage
\end{frame}

\begin{frame}
  \frametitle{Inhaltsübersicht Vorkurs}
  \begin{itemize}
  {\color{gray}
    {\color{black}
    \item {Tag 1: Variablen, Datentypen, Konvertierungen, Arithmetik, Netbeans, Einführung Debugging}
    }
    \item {Tag 2: Boolesche Ausdrücke, Kommentare, If-Abfragen, Switch-Case, Weiterführung Debugging}
    \item {Tag 3: Arrays, (Do-)While-Schleife, For-Schleifen, Weiterführung Debugging}
    \item {Tag 4: (statische) Methoden, Attribute, Ausführung Debugging}
  }
  \end{itemize}
\end{frame}

\section{Ablauf}
\begin{frame}
  \frametitle{Ablauf}
  \begin{itemize}
    \item{09:30 Vorstellung der Lösungen des Vortages}
    \item{ab 10:00 Vorlesung}
    \item{90 min Mittagspause}
    \item{gegen 12:30 / 13:00 Übungen}
  \end{itemize}
\end{frame}

\section{Technologie}
\begin{frame}
  \frametitle{Technologie}
    \begin{tikzpicture}
    \tikzstyle{pfeil} = [rotate=-90, fill=FSIblue,color=FSIblue,single arrow,draw, minimum height=1.5cm,minimum width=1.5cm]
    \tikzstyle{beschriftung} = [node distance=0.7cm, right]
    \tikzstyle{box} = [rounded corners=8pt, minimum height=1.5cm,minimum width=0.9\textwidth,draw,fill=white]
      \node at (0,-3) [minimum height=7.5cm,minimum width=\textwidth,fill=none]{}; 
      \node<2-> (A) at (0,-1.3) [pfeil]{};
      \node<3-> (B) at (0,-4.3) [pfeil]{};
      \node<1-> at (0,0) [box]{Quellcode (Java)};
      \node<2-> at (0,-3) [box]{Zwischencode (Bytecode)};
      \node<3-> at (0,-6) [box]{Maschinencode};
      \node<2-> [right of=A, beschriftung] {Compiler};
      \node<3-> [right of=B, beschriftung] {Laufzeitumgebung (JVM)};
    \end{tikzpicture}
\end{frame}

\section{Variablen}
\begin{frame}
  \frametitle{Variablen}
  \begin{itemize}
    \item{Speicher für Werte, die sich ändern können}
    \item{Primitive Datentypen}
      \begin{itemize}
        \item{Ganzzahlen}
        \item{Kommazahlen}
        \item{Wahrheitswerte}
        \item{Zeichen}
      \end{itemize}
    \item{Referenzdatentypen}
      \begin{itemize}
        \item{Zeichenketten}
      \end{itemize}
  \end{itemize}
\end{frame}

\subsection{Datentypen}
\begin{frame}
  \frametitle{Datentypen}
  \begin{itemize}
    \item{Ganzzahlen}
    \begin{itemize}
      \item{\keyWord{byte} (8 Bit / 1 Byte)}
      \item{\keyWord{short} (16 Bit / 2 Byte)}
      \item{\keyWord{int} (32 Bit / 4 Byte)}
      \item{\keyWord{long} (64 Bit / 8 Byte)}
    \end{itemize}
    \item{Kommazahlen (Gleitkommazahlen)}
    \begin{itemize}
      \item{\keyWord{float} (32 Bit / 4 Byte)}
      \item{\keyWord{double} (64 Bit / 8 Byte)}
    \end{itemize}
  \end{itemize}
  \begin{itemize}
    \item{Unterscheiden sich jeweils nur in ihrem Wertebereich}
  \end{itemize}
\end{frame}

\begin{frame}[fragile]
  \frametitle{Datentypen}
  \begin{itemize}
    \item{Wahrheitswerte}
    \begin{itemize}
      \item{\keyWord{boolean}}
      \item{\keyWord{true} oder \keyWord{false}}
    \end{itemize}
  \end{itemize}
  \begin{itemize}
    \item{1 Zeichen (keine Zeichenkette)}
    \begin{itemize}
      \item{\keyWord{char}}
      \item{2 Byte lang}
      \item{Darstellung als 16-Bit-Unicode-Wert}
    \end{itemize}
  \end{itemize}

  \begin{itemize}
    \item Zeichenketten
    \begin{itemize}
      \item \keyWord{String}
      \item Referenzdatentyp
    \end{itemize}
  \end{itemize}
\end{frame}

\subsection{Wertebereiche}
\begin{frame}[fragile]
  \frametitle{Variablen - Wertebereiche}
  \begin{center}
    \begin{tabular}{l|c|c|l}
      \textbf{Type} & \multicolumn{2}{c|}{\textbf{Länge}} & \textbf{Wertebereich} \\
      & \textbf{Byte} &\textbf{Bit}  & \\
      \hline
      boolean & - &  - & true oder false \\
      char    & 2 & 16 & Unicode Zeichen \\
      byte    & 1 &  8 & $-128$ bis 127 \\
      short   & 2 & 16 & $-32768$ bis 32767 \\
      int     & 4 & 32 & $-2.147.483.648$ bis 2.147.483.647 \\
      long    & 8 & 64 & $-2^{63}$ bis $2^{63}-1$\\
      float   & 4 & 32 & $\pm1,4E-45$ bis $\pm3,4E+38$\\
      double  & 8 & 64 & $\pm4,9E-324$ bis $\pm1,7E+308$\\
    \end{tabular}
  \end{center}
\end{frame}

\section{Variablennamen}
\begin{frame}[fragile]
  \frametitle{Variablennamen}
  \begin{itemize}
    \item Vorgaben
    \begin{itemize}
      \item So MÜSSEN Namen sein, sonst gibt es Compiler-Fehler
      \item Erlaubte Zeichen: Buchstaben, Zahlen und \_
      \item Erstes Zeichen darf keine Zahl sein
    \end{itemize}
    \item Gesperrte Namen
      \begin{itemize}
        \item z.B. \keyWord{true}, \keyWord{false}, \keyWord{new}
      \end{itemize}
  \end{itemize}
\end{frame}

\begin{frame}
  \frametitle{Variablennamen}
  \begin{itemize}
    \item Konventionen
    \begin{itemize}
      \item So SOLLTEN Namen sein, aber der Compiler würde es auch sonst kompilieren
      \item sinnvolle, aussagekräftige Namen
      \item keine Abkürzungen
      \item Substantive
      \item Nur lateinische Zeichen, Zahlen und \_
      \begin{itemize}
        \item KEIN ä, ö, ü, ß, ...
      \end{itemize}
      \item camelCase-Schreibweise
    \end{itemize}
  \end{itemize}
\end{frame}

\section{Verwendung}
\begin{frame}
  \begin{center}
    \Huge{zur Verwendung}
  \end{center}
\end{frame}

\subsection{Deklaration und Wertzuweisung}
\begin{frame}[fragile]
  \frametitle{Deklaration}
  \begin{itemize}
  \item Bekanntmachen der Variablen:
    \begin{lstlisting}[morekeywords={type}]
type name;
    \end{lstlisting}
  \item Beispiele:
    \begin{lstlisting}
int age;
char gender;
boolean isStudent;
    \end{lstlisting}
  \end{itemize}
\end{frame}

\begin{frame}[fragile]
  \frametitle{Wertzuweisung}
  \begin{itemize}
  \item Der Variablen einen Wert zuweisen:
    \begin{lstlisting}[morekeywords={type}]
name = wert;
    \end{lstlisting}
  \item Die Variable muss deklariert worden sein
  \item Beispiele:
    \begin{lstlisting}[morekeywords={String}]
int age; age = 20;
float balance; balance = 4.2f;
char gender; gender = 'm';
String name; name = "Douglas Adams";
    \end{lstlisting}
  \end{itemize}
\end{frame}

\begin{frame}[fragile]
  \frametitle{Deklaration und Wertzuweisung}
  \begin{itemize}
  \item Wert direkt beim Deklarieren auch zuweisen:
    \begin{lstlisting}[morekeywords={type}]
type name = value;
    \end{lstlisting}
  \item Beispiele:
    \begin{lstlisting}[morekeywords={String}]
double average = -5.2;
boolean isStudent = true;
String answer = "42";
    \end{lstlisting}
  \end{itemize}
\end{frame}

\subsection{Ausgabe}
\begin{frame}[fragile]
  \frametitle{Ausgabe}
  \begin{itemize}
    \item Sonst würde es nachher ziemlich langweilig
    \begin{lstlisting}[morekeywords={type}]
System.out.println(ausgabe);
System.out.print(ausgabe);
    \end{lstlisting}
    \item Beispiele:
    \begin{lstlisting}[morekeywords={type}]
System.out.println("Hallo Welt");

String name = "Welt";
System.out.print("Hallo ");
System.out.print(name);
System.out.println();
\end{lstlisting}
  \end{itemize}
\end{frame}

\section{Arithmetik}
\begin{frame}[fragile]
  \frametitle{Arithmetik}
  \begin{center}
  \begin{tabular}{l|c|c}
    \textbf{Bezeichnung} & \textbf{Operator}& \textbf{Anwendung}\\
    \hline
    Addition       & $+$  & a + b   \\
    Subtraktion    & $-$  & a $-$ b \\
    Multiplikation & *    & a * b   \\
    Division       & /    & a / b   \\
    Inkrement      & ++   & a++     \\
    Dekrement      & $--$ & a$--$   \\
    Modulo         & \%   & a \% b  \\
  \end{tabular}
  \end{center}

  \begin{itemize}
    \item[] Ergebnis kann Variablen zugewiesen werden:
    \begin{lstlisting}[morekeywords={type}]
int result = 5 + 2;
double division = 3.5 / (result - 1);
    \end{lstlisting}
  \end{itemize}
\end{frame}

\subsection{Modulo}
\begin{frame}[fragile]
  \frametitle{Modulo (Restwertberechnung)}
  \begin{itemize}
    \item Das Ergebnis des Modulo ist der Rest der Division:
    \begin{lstlisting}
26 / 5 =  5 Rest 1    =>    26 % 5 = 1

30 / 2 = 15 Rest 0    =>    30 % 2 = 0
    \end{lstlisting}
  \end{itemize}
\end{frame}

\subsection{In- bzw. Dekrement}
\begin{frame}[fragile]
  \frametitle{In- bzw. Dekrement}
  \begin{itemize}
    \item Erhöht bzw. erniedrigt den Wert einer Variablen um 1.
    \begin{lstlisting}
int x = 42;
x++; 
x--;
    \end{lstlisting}
\item[] Welchen Wert beinhaltet x? \uncover<2->{ x = 42}
  \end{itemize}
\end{frame}

\begin{frame}[fragile]
\frametitle{Post- bzw. Präinkrement sind zu beachten}
\begin{itemize}
    \begin{lstlisting}
int x = 23;
System.out.println(++x);
\end{lstlisting}
\item[] Wie lautet die Ausgabe und welchen Wert hat x? \uncover<2->{Ausgabe: \textbf{24}; x = 24 }
\item[]\quad
\begin{lstlisting}
int y = 23;
System.out.println(y++);
    \end{lstlisting}
\item[] Wie lautet die Ausgabe und welchen Wert hat y? \uncover<3->{Ausgabe: \textbf{23}; y = 24 }
  \end{itemize}
\end{frame}

\subsection{Integer-Division}
\begin{frame}[fragile]
  \frametitle{Integer-Division}
  \begin{itemize}
    \begin{lstlisting}
int x = 7;

int y = 2;

double z = x / y;
    \end{lstlisting}
    \item[] Welchen Wert beinhaltet z? \qquad \pause z = 3.0
  \end{itemize}
\end{frame}

\subsection{Verknüpfung von Zeichenketten}
\begin{frame}[fragile]
  \frametitle{Verknüpfung von Zeichenketten}
  \begin{itemize}
    \item Verknüpfung durch den +-Operator
    \begin{lstlisting}[morekeywords={type,String}]
String name = "Hallo, " + "Welt";
    \end{lstlisting}
    \item auch gemischt mit Zahlen möglich
    \begin{lstlisting}
int x = 5;
String text = "x hat den Wert " + x;
    \end{lstlisting}
    \item Ausgabe:
    \begin{lstlisting}
System.out.println("x ist " + x);
System.out.print("Hallo, " + "Student");
    \end{lstlisting}
  \end{itemize}
\end{frame}

\section{Konvertierung}
\begin{frame}[fragile]
  \frametitle{Explizite Konvertierung}
  \begin{itemize}
    \item Variablenwerte können umgewandelt werden 
    \begin{itemize}
      \item explizites "`Casten"'
    \end{itemize}
    \begin{lstlisting}
int x = 42;
short y = (short)x;
    \end{lstlisting}
    \item[] Welchen Wert beinhaltet y? \qquad \uncover<2->{ y = 42}
    \begin{lstlisting}
double a = 512.6;
int b = (int)a;
    \end{lstlisting}
    \item[] Welchen Wert beinhaltet b? \qquad \uncover<3->{ b = 512}
  \end{itemize}
\end{frame}

\begin{frame}[fragile]
  \frametitle{Implizite Konvertierung}
  \begin{itemize}
    \item[] Einige Typen können ohne Probleme in andere umgewandelt werden\\
\keyWord{byte \ra short \ra int \ra long \ra float \ra double}
    \begin{lstlisting}
int x = 42;
float y = (float)x;
    \end{lstlisting}
    ist äquivalent zu:
    \begin{lstlisting}
int x = 42;
float y = x;
    \end{lstlisting}
  \end{itemize}
\end{frame}

\begin{frame}[fragile]
  \frametitle{Zurück zum Divisionsproblem}
  \begin{itemize}
    \begin{lstlisting}
int x = 7;

int y = 2;

double z = x / y;
    \end{lstlisting}
    \item Bei Rechnungen wird in den bestmöglichen Typen gecastet
    
    \item[]\keyWord{byte \ra short \ra int \ra long \ra float \ra double}
    \item So funktioniert es:
    \begin{lstlisting}
double z = (double) x / y;
    \end{lstlisting}    
  \end{itemize}
\end{frame}

\section{Kommentare}
\begin{frame}[fragile]
	\frametitle{Kommentare}
	\begin{itemize}
	\item Wird verwendet um Code von der Verwendung auszunehmen oder Kommentare zu hinterlassen. Wenn wir euch auffordern etwas auszukommentieren reden wir hiervon.
	\item Mehrzeilige Kommentare:
	\begin{lstlisting}
	/**
	* Dashier ist alles Kommentar.	
	* int Zahl;
	* char Buchstabe;	
	*/
	\end{lstlisting}
	\item Einzeilige Kommentare:
	\begin{lstlisting}
	int Zahl; //Hier beginnt der Kommentar.
	\end{lstlisting}
	\end{itemize}
\end{frame}

\section{Demonstration NetBeans (Debugging)}
\begin{frame}[fragile]
  \begin{center}
    \Huge{Demonstration NetBeans \\ (Debugging)}
  \end{center}
\end{frame}

\end{document}
